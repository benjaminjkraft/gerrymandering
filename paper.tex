\documentclass[12pt]{article}

\usepackage{fancyhdr,hyperref,setspace,booktabs,amsmath}
\usepackage[pdftex]{graphicx}
\usepackage[margin=1in,includehead]{geometry}

\setcounter{secnumdepth}{0}

\usepackage[backend=bibtex,style=numeric,maxnames=99,isbn=false,url=true]{biblatex}
\renewbibmacro{in:}{}
\addbibresource{paper.bib}

\pagestyle{fancy}
\lhead{Miriam Gershenson, Ben Kraft, and Ryan Lau}
\rhead{6.S897/17.S952 Final Paper}

\begin{document}

  \begin{center}
    \LARGE Comparing Metrics of Gerrymandering
    \vspace{-0.5em}
  \end{center}
  \nocite{*}
  \doublespacing{}

  \section{Introduction}
  
  \section{Background}\label{background}
  
  \subsection{Measuring Compactness}

  When people think of measuring gerrymandering, the first kind of metric that comes to mind is generally one that tries to measure the compactness of a district.  In fact, there are a number of ways of measuring compactness; some are better than others, but many measure different types of compactness.  There are three major types: those that measure the dispersion of the district from its center, those that measure some form of boundary complexity, and those that are based on population and population density.

  The very simplest metrics for compactness simply measure the length and width of the district as compared to that of a circle or rectangle of the same area.  There are several variations on this method, but they are all fairly crude.  Other metrics compare the area of the district to the area of the smallest circumscribing circle, or the largest inscribed circle, or to the convex hull of the district -- the smallest convex shape containing the entire district.  Niemi~et~al.~\cite{niemi} list a number of variations on these metrics.  An alternate method of computing dispersion is by computing the moment of inertia of the area of the district.  The concept is borrowed from physics, and is defined as
  \[I = \iint ((x-x_0)^2 + (y-y_0)^2) dx dy \]
  where $(x_0, y_0)$ is the areal center of mass of the district.  More dispersed districts have higher moments of inertia, because the average square distance of a point in the district to the center is larger.  In order to ensure that the metric does not depend on the size of the state, and to make it range from 0 to 1 with 0 being the least compact, we in fact use the measure $A/\sqrt{2\pi I}$.  This metric seems to have been first proposed by Kaiser.~\cite{kaiser}

  These dispersion measures capture well the extent to which a district is spread out more than necessary -- barbell- and snake-shaped districts score poorly.  However, districts with very complex boundaries aiming to pick up exactly the voters who will advantage one party can still be fairly concentrated close to their centers.  These districts are better detected by measures of boundary complexity.  The usual way to measure this is the perimeter.  In order to account for the varying sizes of districts and states, one can compare the perimeter to that of a circle of the same area, by using the metric $4\pi A/P^2$; several variations are given by Horn.~\cite{horn}  There are some other metrics for boundary complexity which attempt to account for the complexity of existing political boundaries, such as one given by Schwartzberg~\cite{schwartzberg} that measures distances between intersections of political divisions.

  Lastly, since ultimately it is not the geography that matters, but the people that live in it, we can modify some of these metrics to take into account population.  Those that compare the area of the district to that of the smallest circumscribing circle or that of the convex hull of the district may simply compare population instead of area, as proposed by Hofeller and Grofman.~\cite{hofeller}.  Alternately, instead of computing the area-weighted moment of inertia, one can compute the population-weighted moment of inertia, which is defined as
  \[I_P = \iint ((x-x_0)^2 + (y-y_0)^2) \rho dx dy \]
  where $\rho$ is the population density.  In practice, since we do not know the location of each individual in the district, we may simply assume that each person lives at the center of the census tabulation block in which they were counted.  Then the formula becomes
  \[I_P = \sum_{b} ((x_b-x_0)^2 + (y_b-y_0)^2) P_b\]
  where the sum is taken over census tabulation blocks $b$, with $(x_b, y_b)$ the areal center of mass of the census block, and $P_b$ its population.  This measure was first proposed by Weaver.~\cite{weaver}

  \section{Model}

  \section{Methods}

  \subsection{Computing Compactness}

  We used a number of metrics to measure the compactness of districts.  To compute them, we used the geographic data from the U.S.~Census's TIGER/Line database~\cite{censustiger} on the shapes of congressional districts and census blocks, and where applicable, population data from the 2010~Census~\cite{census2010}.

  The simplest metric we computed was $4\pi A/P^2$.  The only complication to computing this metrics was the handling of coastlines, for which the district boundary is generally drawn a few miles offshore.  Rather than try to resolve this issue by clipping the district to the coastline, and therefore potentially increasing the perimeter of the district due to the complexity of the coastline, we simply computed the perimeter and area as drawn; we think the error introduced this way is likely smaller than that introduced by attempting to correct for it.
  
  The population moment of area was also simple to compute, since the TIGER/Line data approximates each district as a polygon.  Again, we ignored the issue of clipping to the coastline to avoid introducing more error.  Using census block population data, we also computed the population moment of inertia of each district.  We used the method described in Section~\ref{background} to deal with the limited granularity of the population data.

  %TODO: Miriam fill in convex hull and circumscribed circle metrics here.
  %TODO: possibly reorganize this subsection once Miriam adds her parts

  With the non-population-based metrics especially, Alaska and Hawaii would present particular difficulties, since they include a number of widely dispersed islands.  However, Alaska has only one representative, and Hawaii only two, so since our analysis focused on states with at least three districts, we did not attempt to resolve these issues.

  \section{Analysis}

  \section{Dicussion}

  \section{Conclusion}
  
  \singlespacing{}

  \printbibliography{}

\end{document}
