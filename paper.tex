\documentclass[12pt]{article}

\usepackage{fancyhdr,hyperref,setspace,booktabs,amsmath}
\usepackage[pdftex]{graphicx}
\usepackage[margin=1in,includehead]{geometry}

\setcounter{secnumdepth}{0}

\usepackage[backend=bibtex,style=numeric,maxnames=99,isbn=false,url=true]{biblatex}
\renewbibmacro{in:}{}
\addbibresource{paper.bib}

\pagestyle{fancy}
\lhead{Miriam Gershenson, Ben Kraft, and Ryan Lau}
\rhead{6.S897/17.S952 Final Paper}

\begin{document}

  \begin{center}
    \LARGE Comparing Metrics of Gerrymandering
    \vspace{-0.5em}
  \end{center}
  \nocite{*}
  \doublespacing{}

  \section{Introduction}
  
  \section{Background}\label{background}
  
  \subsection{Measuring Compactness}

  When people think of measuring gerrymandering, the first kind of metric that comes to mind is generally one that tries to measure the compactness of a district.  In fact, there are a number of ways of measuring compactness; some are better than others, but many measure different types of compactness.  There are three major types: those that measure the dispersion of the district from its center, those that measure some form of boundary complexity, and those that are based on population and population density.

  The very simplest metrics for compactness simply measure the length and width of the district as compared to that of a circle or rectangle of the same area.  There are several variations on this method, but they are all fairly crude.  Other metrics compare the area of the district to the area of the smallest circumscribing circle, or the largest inscribed circle, or to the convex hull of the district -- the smallest convex shape containing the entire district.  Niemi~et~al.~\cite{niemi} list a number of variations on these metrics.  An alternate method of computing dispersion is by computing the moment of inertia of the area of the district.  The concept is borrowed from physics, and is defined as
  \[I = \iint ((x-x_0)^2 + (y-y_0)^2) dx dy \]
  where $(x_0, y_0)$ is the areal center of mass of the district.  More dispersed districts have higher moments of inertia, because the average square distance of a point in the district to the center is larger.  In order to ensure that the metric does not depend on the size of the state, and to make it range from 0 to 1 with 0 being the least compact, we in fact use the measure $A/\sqrt{2\pi I}$.  This metric seems to have been first proposed by Kaiser.~\cite{kaiser}

  These dispersion measures capture well the extent to which a district is spread out more than necessary -- barbell- and snake-shaped districts score poorly.  However, districts with very complex boundaries aiming to pick up exactly the voters who will advantage one party can still be fairly concentrated close to their centers.  These districts are better detected by measures of boundary complexity.  The usual way to measure this is the perimeter.  In order to account for the varying sizes of districts and states, one can compare the perimeter to that of a circle of the same area, by using the metric $4\pi A/P^2$; several variations are given by Horn.~\cite{horn}  There are some other metrics for boundary complexity which attempt to account for the complexity of existing political boundaries, such as one given by Schwartzberg~\cite{schwartzberg} that measures distances between intersections of political divisions.

  Lastly, since ultimately it is not the geography that matters, but the people that live in it, we can modify some of these metrics to take into account population.  Those that compare the area of the district to that of the smallest circumscribing circle or that of the convex hull of the district may simply compare population instead of area, as proposed by Hofeller and Grofman.~\cite{hofeller}.  Alternately, instead of computing the area-weighted moment of inertia, one can compute the population-weighted moment of inertia, which is defined as
  \[I_P = \iint ((x-x_0)^2 + (y-y_0)^2) \rho dx dy \]
  where $\rho$ is the population density.  In practice, since we do not know the location of each individual in the district, we may simply assume that each person lives at the center of the census tabulation block in which they were counted.  Then the formula becomes
  \[I_P = \sum_{b} ((x_b-x_0)^2 + (y_b-y_0)^2) P_b\]
  where the sum is taken over census tabulation blocks $b$, with $(x_b, y_b)$ the areal center of mass of the census block, and $P_b$ its population.  This measure was first proposed by Weaver.~\cite{weaver}
  
    \subsection{Partisan Symmetry and Vote Efficiency}
 While geographic compactness is a widely-considered metric for compactness, others have focused on actual results for determining unacceptable redistricting plans.  Over the last few years, the courts have increasingly rejected a variety of rationales cited to claim illegal gerrymandering, including "partisan intent",  "minority party entrenchment", and "lack of proportionality".  However, in \textit{League of United Latin American Citizens (LULAC) v. Perry, Governor of Texas et al.}, (2006), the majority recognized that it would accept a lack of \textit{symmetry} as an acceptable argument for determining unconstitutional gerrymandering. \\
  
  A given plan has symmetry if "similary-situated parties are treated equally", thus expecting that an even split in votes across two parties would have an even split in seats. Furthermore, if a plaintiff challenging a redistricting plan can demonstrate partisan intent and asymmetry, it would be sufficient for the burden of proof to shift to the state.  Nevertheless, while acknowledging the viability of symmetry as a metric to determine a discriminatory impact on voters, the court failed to establish an accepted standard or test to determine illegal gerrymandering.  Existing measures at the time, as calculated by social scientists, used hypothetical models to predict how election outcomes would turn out, given certain shifts in the electorate and certain vote shares in a given election. The court rejected the notion of using hypothetical models for a standard of constitutionality. \cite{LULAC v. Perry, 2006} \\
  
  With this in mind,  as an alternative to other existing metrics, two legal scholars recently proposed measuring \textit{vote efficiency} as a way to quantify partisan symmetry and partisan gerrymandering \cite{Stephanopolous & McGhee}.  Vote efficiency is a measure of how well a given vote applied towards improving political representation.  In the case that a voter's desired candidate wins, any vote beyond what is necessary to win (in the simple case, 50\% + 1 vote) is wasted.  In the case when a voter's desired candidate loses, all votes for the losing candidate are wasted. Redistricting plans seeking a partisan bias will lead to fewer wasted votes for the favored party, while increasing wasted votes for the disadvantaged party. \\
  
  Vote efficiency helps to overcome several of the deficiencies expressed by the Supreme Court in \textit{LULAC}.  First, by using actual election data, rather than hypothetical models, the Court can establish a definite model for determining partisan bias.  Furthermore, it will allow the court to establish a test threshold for considering when a plan becomes unconstitutional.  This measurement will be explained precisely in the Methods section.

  \section{Model}
  In this paper, we will conduct an analysis on the correlation between different measures of compactness and political bias, as a method for determining the overall asymmetry of a given political area.  Furthermore, we will examine the political processes involved in drawing those areas and compare the processes with the levels of gerrymandering exhibited.\\
  
  We expect to see the following in our results:
  \begin{itemize}
  \item Compactness and political bias/efficiency will correlate well, with population metrics and dispersion from center metrics performing more strongly than simple area and perimeter metrics 
  \item Independent and bi-partisan commissions will outperform partisan legislative and advisory processes in producing less-gerrymandered districting plans.
  \end{itemize}
  
  We predict these outcomes on the following basis: non-compact districts are oftentimes designed in such a way that extensive work to include wide swaths of voters will lead to districts where people are spread out over a wide area or in a few very dense spaces.  Furthermore, these districts are designed to create partisan bias, not partisan fairness, and thus will correlate with the voter efficiency.  Also, we expect to find that partisan legislative process will more likely produce gerrymandered districts, as the lack the political checks and balances than an independent or bi-partisan political body would ensure.  While we do not expect all partisan legislative processes to have gerrymandered districts, we still expect that all gerrymandered districts will be created by partisan legislative processes.

  \section{Methods}

  \subsection{Computing Compactness}

  We used a number of metrics to measure the compactness of districts.  To compute them, we used the geographic data from the U.S.~Census's TIGER/Line database~\cite{censustiger} on the shapes of congressional districts and census blocks, and where applicable, population data from the 2010~Census~\cite{census2010}.

  The simplest metric we computed was $4\pi A/P^2$.  The only complication to computing this metrics was the handling of coastlines, for which the district boundary is generally drawn a few miles offshore.  Rather than try to resolve this issue by clipping the district to the coastline, and therefore potentially increasing the perimeter of the district due to the complexity of the coastline, we simply computed the perimeter and area as drawn; we think the error introduced this way is likely smaller than that introduced by attempting to correct for it.
  
  The population moment of area was also simple to compute, since the TIGER/Line data approximates each district as a polygon.  Again, we ignored the issue of clipping to the coastline to avoid introducing more error.  Using census block population data, we also computed the population moment of inertia of each district.  We used the method described in Section~\ref{background} to deal with the limited granularity of the population data.

  %TODO: Miriam fill in convex hull and circumscribed circle metrics here.
  %TODO: possibly reorganize this subsection once Miriam adds her parts

  With the non-population-based metrics especially, Alaska and Hawaii would present particular difficulties, since they include a number of widely dispersed islands.  However, Alaska has only one representative, and Hawaii only two, so since our analysis focused on states with at least three districts, we did not attempt to resolve these issues.

  \subsection {Computing Voter Efficiency}
  We calculate voter efficiency using data from a variety of sources.  The US Census Data, along with the Harvard Election Data Archive presents a great resource for Federal Election results.  State-by-state elections are best sourced from the respective state agencies that govern the election, such as the Secretary of State's office or Board of Elections.  To calculate voter efficiency, we first go back to the definition of wasted votes.  This measure of voter efficiency will only analyze two-party, single-winner elections.  Thus, for a given party $p$, their opposing party $o$ and a given district $i$, the number of wasted votes  if party $p$ wins ($V_{p} > V_{o}$)  is given by 
    \[ V_{wpi} = V_{pi} - \frac{V_{pi}+V_{oi}}{2}\]
    
    as all votes beyond 50\% are wasted. Otherwise, if party $p$ loses the district, then
    \[V_{wpi} = V_{pi}\]
    as all votes that do not go towards electing a candidate are wasted.\\
    
    While this calculates the wasted votes for an individual district, vote efficiency is calculated at the aggregate state level.  The vote efficiency $V_{Ep}$ for party $p$ is then calculated as follows:
     \[V_{Ep} = \frac{\sum_{i=1}^{n}(V_{wpi}-V_{woi})}{\sum_{i=1}^{n}(V_{pi}+V_{oi})}\]
     
     Stephanopolous and McGhee also propose that $V_{pi}+V_{oi}$ remains constant, as the size of any district in a given state cannot vary significantly, thus requiring equality.  Under this assumption, where the total votes in any given district are constant at $V_{t} = V_{pi}+V_{oi}$, and party $p$ wins $m$ of $n$ seats, the above equation simplifies to:
     \[V_{Ep} = (\frac{m}{n}-0.5) - 2*(\frac{\sum^{n}V_{pi}}{\sum^n{V_t}}-0.5)\]
  
  The first term, $(\frac{m}{n}-0.5)$ represents the "Seat Share" or the proportion of seats that were won by party $p$ in excess (or deficient) of half the seats.  The second term represents the "Vote Share" or the proportion of the vote that was won by party $p$ in excess of half. The 0.5 adjustment terms are there so that a positive $V_{Ep}$ indicates that the given party has an advantage due to partisan gerrymandering and a disadvantage with a negative $V_{Ep}$.  Furthermore, it represents the proportion of seats lost due to this disadvantage.\\
  
  Overall, the voter share/seat share metric is much easier to calculate, as nothing needs to be known about the individual districts, but rather the totals of the overall election.  However, as we will expose in the analysis and discussion sections, the assumption that all districts have the same total votes is wrong due to turnout variance and creates issues when analyzing the data.\\
  

  \section{Analysis}

  \section{Discussion}

  \section{Conclusion}
  
  \subsection{Court Test Threshold}
  10\% for areas w/at least 3 districts
  \subsection{Policy Recommendations}
  Use nonpartisan methods
  
  \singlespacing{}

  \printbibliography{}

\end{document}
