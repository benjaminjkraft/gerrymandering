\documentclass[12pt]{article}

\usepackage{fancyhdr,hyperref,setspace,booktabs}
\usepackage[pdftex]{graphicx}
\usepackage[margin=1in,includehead]{geometry}

\setcounter{secnumdepth}{0}

\usepackage[backend=bibtex,style=numeric,maxnames=99,isbn=false,url=true]{biblatex}
\renewbibmacro{in:}{}
\addbibresource{paper.bib}

\pagestyle{fancy}
\lhead{Miriam Gershenson, Ben Kraft, and Ryan Lau}
\rhead{6.S897/17.S952 Final Paper}

\begin{document}

  \begin{center}
    \LARGE Comparing Metrics of Gerrymandering
    \vspace{-0.5em}
  \end{center}
  \nocite{*}
  \doublespacing{}

  \section{Introduction}
 
  \section{Background}

  We cite some people~\cite{chenrodden}.
  
  \subsection{Partisan Symmetry and Vote Efficiency}
 While geographic compactness is a widely-considered metric for compactness, others have focused on actual results for determining unacceptable redistricting plans.  Over the last few years, the courts have increasingly rejected a variety of rationales cited to claim illegal gerrymandering, including "partisan intent",  "minority party entrenchment", and "lack of proportionality".  However, in \textit{League of United Latin American Citizens (LULAC) v. Perry, Governor of Texas et al.}, (2006), the majority recognized that it would accept a lack of \textit{symmetry} as an acceptable argument for determining unconstitutional gerrymandering. \\
  
  A given plan has symmetry if "similary-situated parties are treated equally", thus expecting that an even split in votes across two parties would have an even split in seats. Furthermore, if a plaintiff challenging a redistricting plan can demonstrate partisan intent and asymmetry, it would be sufficient for the burden of proof to shift to the state.  Nevertheless, while acknowledging the viability of symmetry as a metric to determine a discriminatory impact on voters, the court failed to establish an accepted standard or test to determine illegal gerrymandering.  Existing measures at the time, as calculated by social scientists, used hypothetical models to predict how election outcomes would turn out, given certain shifts in the electorate and certain vote shares in a given election. The court rejected the notion of using hypothetical models for a standard of constitutionality. \cite{LULAC v. Perry, 2006} \\
  
  With this in mind,  as an alternative to other existing metrics, two legal scholars recently proposed measuring \textit{vote efficiency} as a way to quantify partisan symmetry and partisan gerrymandering \cite{Stephanopolous & McGhee}.  Vote efficiency is a measure of how well a given vote applied towards improving political representation.  In the case that a voter's desired candidate wins, any vote beyond what is necessary to win (in the simple case, 50\% + 1 vote) is wasted.  In the case when a voter's desired candidate loses, all votes for the losing candidate are wasted. Redistricting plans seeking a partisan bias will lead to fewer wasted votes for the favored party, while increasing wasted votes for the disadvantaged party. \\
  
  Vote efficiency helps to overcome several of the deficiencies expressed by the Supreme Court in \textit{LULAC}.  First, by using actual election data, rather than hypothetical models, the Court can establish a definite model for determining partisan bias.  Furthermore, it will allow the court to establish a test threshold for considering when a plan becomes unconstitutional.  This measurement will be explained precisely in the Methods section.
  
  
  \section{Model}

  \section{Methods}

  \section{Analysis}

  \section{Discussion}

  \section{Conclusion}
  
  \singlespacing{}

  \printbibliography{}

\end{document}
